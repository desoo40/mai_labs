\documentclass[12pt]{article}

\usepackage{fullpage}
\usepackage{multicol,multirow}
\usepackage{tabularx}
\usepackage{ulem}
\usepackage[utf8]{inputenc}
\usepackage[russian]{babel}


\begin{document}

\section*{Лабораторная работа №\,1 по курсу дискрeтного анализа: сортировка за линейное время}

Выполнил студент группы 8О-207 МАИ \textit{Сорокин Денис}.

\subsection*{Условие}

Разработать программу, осуществляющую ввод пар «ключ-значение» 
\textit{(ключ  - Числа от 0 до $2^{64}$ - 1, значение - числа от 0 до $2^{64}$ - 1)} 
их упорядочивание по возрастанию \textit{поразрядной сортировкой} 
за линейное время и вывод отсортированной последовательности.


\subsection*{Метод решения}
\begin{enumerate}
\item Создаем динамический массив, в котором будут храниться классы с парами "ключ-значениe".
\item Считываем со стандартного потока эти пары, пока не встретимся конец ввода.
\item \begin{enumerate}
	\item Если массив полностью заполнился, и добавлять элемнет некуда - увеличиваем размер массива в два раза и запысываем наш элемент.
	\item Иначе просто добавляем элемент в массив.
\end{enumerate}
\item Данный массив отправляем на поразрядную сортировку.
\item Каждый разряд ключа сортируем сортировкой подсчётом.
\item Отсортированный массив выводим через стандартный поток парами "ключ-значение".
\end{enumerate}
Используемая литература:
\begin{enumerate}
\item \textit{"Лекции по дискретному анализу"} Макаров Н.К.
\item Статья "Цифровая сортировка" на сайте http://neerc.ifmo.ru/wiki
\end{enumerate}
\subsection*{Описание программы}

В данной программе один файл, но есть подразделение на функции:
\begin{itemize}
\item main() - главная функция, в которой осуществляется ввод массива, а так же вызов других функций.
\item RadixSort() - функция, осуществляющая сортировку массива.
\item RadixSortResult() - печатает отсортированный массив.
\item Digit() - возвращает нужный нам разряд из числа.
\item MaxRadix() - возвращает максимальный разряд в массиве.

\end{itemize}

\subsection*{Дневник отладки}
В ходе выполнения были допущены ошибки вот самые яркие из них:

\begin{itemize}
\item Были мелкие недочеты в алгоритме, в результате которых сортировка шла не так, как положено. Правилось все за счет изменения границ в for.
\item Программа падала на тесте без данных. Проблема была решена путем добавления проверки на пустоту в функцию RadixSort().
\item При рефакторинге кода неверно понял подсказку VisualStudio. Студия подсказывала, что слово else лишнее, я же подумал, что можно удалить и тело условия, в следствие чего потом долго пытался разобраться, в чем проблема.
\end{itemize}

\subsection*{Тест производительности}

Все тесты случайно сгенерированны, числа в паре "ключ-значение" не превышают 10000.

\begin{table} [h]

\begin{center}
\begin{tabular}{|c|c|c|}
\hline
& \multicolumn{2}{c|}{Время сортировки, c} \\
\cline{2-3}
\raisebox{1.5ex}[0cm][0cm]{Размер входных данных}
& Поразрядная & STL sort() \\
\hline
100 & 4.2e-05 & 3.3e-05 \\
\hline
1000 &  0.00036 & 0.000311 \\
\hline
10000 & 0.005619 & 0.004896 \\
\hline
100000 & 0.035592 & 0.056523 \\
\hline
1000000 & 0.431176 & 0.542368 \\
\hline
10000000 &  4.03 & 6.17\\
\hline
\end{tabular}
\end{center}
\end{table}

\subsection*{Недочёты}

\begin{itemize}
\item Не удалось вынести считывание массива в отдельную функцию, при вынесении и работе с указателями программа падала на 7 тесте. Отсавляя же считывание внутри функции main(), программа работает корректно.
\item Произошло смешение кода C и C++. Например я использую new (C++) для выделения памяти и realloc (C) для перераспределения.
\end{itemize}

\subsection*{Выводы}

В ходе выполнения работы я приобрел необходимые программисту навыки написания алгоритмов и использования сортировок. Так же набрался опыта в отладке кода.

\end{document}